\documentclass[a4paper, parskip=half]{scrartcl}
\usepackage[margin=10mm]{geometry}

\usepackage{tikz}
\usetikzlibrary{shapes.geometric}

\usepackage{fontspec}

\usepackage{contour}
\usepackage{multicol}
\setlength\columnsep{10mm}

\usepackage{enumitem}
\usepackage{booktabs}

\usepackage{lipsum}

\setkomafont{section}{\setmainfont{Boecklins Universe}\LARGE\center}%{\setmainfont{Limelight}\LARGE\center}%{\setmainfont{Half Full NF}\LARGE\center}%{\setmainfont{Orbitron-Black}\LARGE\center}

% Adjust spacing before and after section headings
\RedeclareSectionCommand[
  runin=false,
  beforeskip=1.0\baselineskip,
  afterskip=-0.0\baselineskip
]{section}

\definecolor{Crimson}{HTML}{DC143C}
\definecolor{Green}{HTML}{00A550}
\definecolor{RichLavender}{HTML}{A76BCF}
\definecolor{RoyalBlue}{HTML}{4169E1}
\definecolor{Copper}{HTML}{7B3F00}

\tikzset{pics/icon/.style n args={2}{code={
\node[inner sep=2mm, draw, line width=0.75mm, fill=white, fill opacity=1.0, rounded corners=2mm, minimum width=1.5cm] {\tikz{
\node[inner sep=0pt] (pc) {\includegraphics[scale=0.125]{Icons/#1}};
\path (pc.south) --++ (0, -0.05cm) node[inner sep=0pt] {\textcolor{black}{\tiny #2}};
}};
}}}

\begin{document}
{
\setmainfont[Scale=2.4]{Boecklins Universe}%{Limelight}%{Half Full NF}%{Orbitron-Black}
\Huge
\begin{center}
%Xenoastrology
\raisebox{-0.2ex}{\tikz{\pic {icon={player_count_icon_black.png}{1-99}};}} \hfill Ast\textcolor{black}{roll}ogy \hfill \raisebox{-0.2ex}{\tikz{\pic {icon={play_time_icon_black.png}{10-15}};}}% \raisebox{-0.3ex}{\includegraphics[height=1.6ex]{Icons/play_time_icon_red.png}}
\end{center}
}

\vspace{-10.5mm}

{
\setmainfont{Ubuntu}%{Glass Antiqua}%{Half Full NF}%{TeX Gyre Schola}
\begin{center}
\input{starfield_diagram.tex}
\end{center}
}

\vspace{-10mm}

\setmainfont{TeX Gyre Schola}%[Scale=1.2]{Glass Antiqua}
\raggedright
\begin{multicols}{2}
\section*{Gameplay}
Give everyone a pencil and a copy of these rules. Find two standard six-sided dice to share.

Play proceeds over a series of rounds. Each round, roll the dice. Then each player should either:
\begin{itemize}[leftmargin=*]
%	\vspace{-1ex}
	\item Choose a new \emph{current star} that matches a die result and is not in a constellation.	
	\item Draw a straight line from their current star to a star that matches a die result and is not in a constellation. Then, update their current star.
\end{itemize}

\textbf{Notes:} A \emph{constellation} is a group of stars that are connected by lines. Each constellation may have at most one star of each number. Lines may not cross other lines or pass through intervening stars.\vfill\null\columnbreak

%Every set of stars that are connected by lines forms a \emph{constellation}. Each constellation may contain at most one star of each number.\vfill\null\columnbreak

\section*{Scoring}
Each round, any players who cannot take an action will receive a \emph{strike} ({\setmainfont{Ubuntu} X}). The game ends when any player receives their third strike.

Players are then awarded points based on the number of stars in each of their constellations according to the following table:

\begin{center}
\begin{tabular}{l rrrrr} \toprule
Stars & 2 & 3 & 4 & 5 & 6 \\
Points & \phantom{1}1 & \phantom{1}3 &\phantom{1}6 & 10 & 15 \\ \bottomrule
\end{tabular}
\end{center}

\vspace{1.0ex}

The player with the highest total score wins.

\vspace{1ex}

\textbf{Design:} Michael Purcell\\
\textbf{Random Seed:} 1209966854\vfill\null
\end{multicols}
\end{document}
